%\documentclass[handout]{beamer}
\documentclass{beamer}
\usepackage{etex} % fixes new-dimension error

\definecolor{BrickRed}{rgb}{0.8, 0.25, 0.33}

%\usetheme{CambridgeUS}%{Copenhagen}%{Frankfurt}%{Singapore}%{CambridgeUS}
\usecolortheme[named=BrickRed]{structure} 
\useoutertheme[subsection=false]{miniframes}
\useinnertheme{circles}
%\useinnertheme[shadow=false]{rounded}
\setbeamertemplate{blocks}[rounded][shadow=false]

\setbeamercovered{transparent} 
\setbeamertemplate{navigation symbols}{} %Remove navigation bar
\setbeamertemplate{footline}[frame number] % add page number
\setbeamercolor{postit}{fg=BrickRed,bg=red!10}
\setbeamercolor{structure}{bg=black!10}
%\setbeamercolor{palette primary}{use=structure,fg=red,bg=green}
%\setbeamercolor{palette secondary}{use=structure,fg=red!75!black,bg=green}
\setbeamercolor{palette tertiary}{use=structure,bg=red!50!black,fg=white}
%\setbeamercolor{palette quaternary}{fg=black,bg=green}
%\setbeamercolor{normal text}{fg=black,bg=white}
%\setbeamercolor{block title alerted}{fg=red,bg=green}
%\setbeamercolor{block title example}{bg=black!10,fg=green}
\setbeamercolor{block body}{bg= black!5}

\setbeamercolor{block title alerted}{bg=red!25}
\setbeamercolor{block body alerted}{bg=red!10}

\setbeamercolor{block title example}{bg={rgb:green,2;black,1;white,5}}
\setbeamercolor{block body example}{bg={rgb:green,2;black,1;white,20}}


%----------------------------------------------------------------------------
\usepackage{graphicx,amsmath}
\usepackage{stmaryrd} % cf. interleave
\usepackage{./macros/myisolatin1}

\usepackage{amscd}
%------ using xy ------------------------------------------------------------
\usepackage{alltt}
%------ using xy ------------------------------------------------------------
\usepackage[all]{xy}
%\def\larrow#1#2#3{\xymatrix{ #3 & #1 \ar[l] _-{#2} }}
\def\larrow#1#2#3{\xymatrix{ #3 & #1 \ar[l] _--{#2} }}
\def\rarrow#1#2#3{\xymatrix{ #1 \ar[r]^-{#2} & #3 }}
\def\arLaw#1#2#3#4#5{
\xymatrix{
        #1      \ar@/^1pc/[rr]^-{#4} &
        #5 &
        #2      \ar@/^1pc/[ll]^-{#3}
}}
\def\arLeq#1#2#3#4{\arLaw{#1}{#2}{#3}{#4}\leq}

%\newcommand{\fdec}[3]{#1:\, #2\, \longrightarrow\, #3}
%------ using pstricks (rnode etc) ------------------------------------------
%\usepackage{pstricks,pst-node,pst-text,pst-3d}
%
%
%%------ using color ---------------------------------------------------------
%
%\newrgbcolor{goldenrod}{.80392 .60784 .11373}
%\newrgbcolor{darkgoldenrod}{.5451 .39608 .03137}
%\newrgbcolor{brown}{.15 .15 .15}
%\newrgbcolor{darkolivegreen}{.33333 .41961 .18431}
%
%
\def\gold#1{{\goldenrod #1}}
\def\dgold#1{{\darkgoldenrod #1}}
%\def\brw#1{{\brown #1}}
\def\dkb#1{{\blue #1}}
\def\tdkb#1{\textbf{\darkblue #1}}
%%\def\gre#1{{\green #1}}
\def\gre#1{{\darkolivegreen #1}}
\def\gry#1{{\gray #1}}
\def\rdb#1{{\red #1}}

\def\laplace#1#2{*\txt{\mbox{ \fcolorbox{black}{myGray}{$\begin{array}{c}\mbox{#1}\\\\#2\\\\\end{array}$} }}}

\def\Lg#1{\mathsf{Lang}(#1)}
\def\Tr#1{\mathsf{Tr}(#1)}
\input{./macros/LSBslides-mac}

\title{
	Introduction to labelled transition systems
	}
\author{Jos\'e Proen\c{c}a}
\institute{HASLab - INESC TEC \\ Universidade do Minho\\ Braga, Portugal}

\date{
\begin{tabular}{c}
\\
    February, 2016
\end{tabular}
}

\begin{document}

\frame[plain]{\titlepage}


\section{LTS -- Basic definitions}
%----------------------------------------------------------------------------------
\begin{slide}{Reactive systems}
\small
\begin{block}{Reactive system}
\caixa{system that computes by reacting to stimuli from its environment along its overall computation}
\end{block}
\begin{itemize}
\item in contrast to sequential systems whose meaning is defined by the results of finite computations, the behaviour of reactive systems is mainly determined by \dkb{interaction} and \dkb{mobility} of \dkb{non-terminating} processes, 
evolving \dkb{concurrently}.
\item \dkb{observation} $\; \equiv\;$ interaction
\item \dkb{behaviour} $\; \equiv\;$ a structured record of interactions
\end{itemize}
\end{slide}

%----------------------------------------------------------------------------------
\begin{slide}{Labelled Transition System}
\small
\begin{block}{Definition}
A LTS over a set $\Act$ of names is a tuple
$\pair{S, \Act,  \rra}$ where
\begin{itemize}
\item $S = \enset{s_0, s_1, s_2, ...}$ is a set of states
%\item  $\downarrow \subseteq S$ is the set of \dkb{terminating} or final states
%\begin{equation*}
%\downarrow s \; \equiv\; s \in \downarrow
%\end{equation*}
\item  $\rra \subseteq S \times \Act \times S$ is the transition relation, often given as an $\Act$-indexed family of 
binary relations 
\begin{equation*}
s \rtran{a} s'\; \equiv\; \pair{s',a,s} \in \rra
\end{equation*}
\end{itemize}
\end{block}
\end{slide}

%----------------------------------------------------------------------------------
\begin{slide}{Labelled Transition System}
\small
\begin{block}{Morphism}
A \dkb{morphism} relating two LTS over  $\Act$, $\pair{S,\Act, \rra}$ and $\pair{S',\Act,  \rra'}$,
is a function $\fdec{h}{S}{S'}$  st
\begin{align*}
s \rtran{a} s' \; \; \; \; \;  \imp & \; \; \; \; \; \apf{h}{s}  \primertran{a} \apf{h}{s'} 
\end{align*}
\end{block}

\begin{center}
\fbox{morphisms \dkb{preserve} transitions}
\end{center}
\end{slide}

%----------------------------------------------------------------------------------
\begin{slide}{Labelled Transition System}
\small
\begin{block}{System}
Given a LTS $\pair{S, \Act, \rra}$, each state $s \in S$ determines a \dkb{system} over 
all states reachable from $s$ and the corresponding restrictions of $\rra$ and $\downarrow$.
\end{block}

\begin{block}{LTS classification}
\begin{itemize}
\item \dgold{deterministic}
\item \dgold{non deterministic}
\item \dgold{finite}
\item \dgold{finitely branching}
\item \dgold{image finite}
\item ...
\end{itemize}
\end{block}

\end{slide}


%----------------------------------------------------------------------------------
\begin{slide}{Reachability}
\small
\begin{block}{Definition}
The reachability relation, $\rra^* \subseteq S \times \Act^* \times S$, is defined inductively
\begin{itemize}
\item $s \reach{\!\!\!\!\!\epsilon} s$ for each $s \in S$, where $\epsilon \in \Act^*$ denotes the empty word;
\item if  $s \rtran{a} s''$  and $s'' \reach{\!\!\!\!\!\sigma} s'$ then $s \reach{\!\!\!\!\!a \sigma} s'$, for $a \in \Act, \sigma \in \Act^*$
\end{itemize}
\end{block}

\begin{block}{Reachable state}
$t \in S$ is \dkb{reachable} from $s \in S$ iff there is a word $\sigma \in \Act^*$ st $s \reach{\!\!\!\!\!\sigma} t$
\end{block}

\end{slide}

%----------------------------------------------------------------------------------
%\begin{slide}{An alternative characterisation}
%\small
%\begin{block}{Coalgebraic  characterization (morphism)}
%A \dkb{morphism} $\fdec{h}{\pair{S, \nx}}{\pair{S', \nx'}}$ is a function 
%$\fdec{h}{S}{S'}$ st the following diagram commutes
%\begin{equation*}
%\xymatrix{
%S \times \Act \ar[d]_-{h \times \id} \ar[r]^-{\nx} & \ppow{S} \ar[d]^-{\ppow{h}}\\
%S' \times \Act \ar[r]^-{\nx'} & \ppow{S'}
%}
%\end{equation*}
%i.e.,
%\begin{equation*}
%\ppow{h} \comp \nx\; \; =\; \; \nx' \comp (h \times \id)
%\end{equation*}
%or, going pointwise,
%\begin{equation*}
%\setdef{\apf{h}{x}}{x \in \apf{\nx}{\pair{s,a}}}\; \;   =\; \;  \apf{\nx'}{\pair{\apf{h}{s},a}} 
%\end{equation*}
%\end{block}
%\end{slide}
%
%%----------------------------------------------------------------------------------
%\begin{slide}{An alternative characterisation}
%\small
%\begin{block}{Coalgebraic  characterization (morphism)}
%A \dkb{morphism} $\fdec{h}{\pair{S, \nx}}{\pair{S', \nx'}}$ 
%\begin{itemize}
%\item \dkb{preseves} transitions:
%\begin{equation*}
%s' \in \apf{\nx}{\pair{s,a}} \imp \apf{h}{s'} \in \apf{\nx'}{\pair{\apf{h}{s},a}} 
%\end{equation*}
%\item \dkb{reflects} transitions:
%\begin{equation*}
%r' \in \apf{\nx'}{\pair{\apf{h}{s},a}} \imp \rcb{\exists}{s' \in S}{s' \in \apf{\nx}{\pair{s,a}}}{r' = \apf{h}{s'}}  
%\end{equation*}
%\end{itemize}
%\end{block}
%\dgold{(why?)}
%\end{slide}
%
%
%%----------------------------------------------------------------------------------
%\begin{slide}{Comparison}
%\small
%
%\begin{itemize}
%\item Both definitions coincide at the \dgold{object} level:
%\begin{equation*}
%\pair{s,a,s'} \in T\; \; \equiv\;\; s' \in \apf{\nx}{\pair{s,a}}
%\end{equation*}
%\item Wrt \dgold{morphisms}, the relational definition is more general, corresponding, in coalgebraic terms to
%\begin{equation*}
%\ppow{h} \comp \nx\; \; \subseteq\; \; \nx' \comp (h \times \id)
%\end{equation*}
%\end{itemize}
%
%
%\pause
%
%%\begin{alertblock}
%\myblock{How can these notions of \dkb{morphism} be used to compare LTS?}
%%\end{alertblock}
%
%
%\end{slide}


\section{Process algebra}

%-------------------------------------------------------------------------------
\newcommand{\transp}[2][35]{\color{fg!#1}#2}

\begin{slide}{Process algebras}
\small

\begin{block}{CCS - Syntax}
\begin{equation*}
\mathcal{P} ~\ni~ P,Q\; ::=\; K ~~|~~ \alpha.P ~~|~~ \sum_{i\in I} P_i
        ~~|~~ \crn P f \transp{~~|~~ P|Q ~~|~~ P\backslash L}
\end{equation*}
%
where
\\- $\alpha \in \Act \transp{\cup \overline\Act \cup \set{\tau}} $,
\\- $K$ s a collection of process names or process contants,
\\- $I$ is an indexing set, and
\\- $L \subseteq \Act \transp{\cup \overline{\Act}}$
\\- $f$ is a function that renames actions
\\- notation: $\cnil \ sum_{i\in\emptyset}P_i$; $P_1+P_2 = \sum_{i\in\set{1,2}}P_i$; $f = [b_1/a_1,\ldots,b_n/a_n]$
\end{block}
\end{slide}

%-------------------------------------------------------------------------------

\begin{slide}{Process algebras}
\small

\begin{block}{Syntax}
\begin{equation*}
\mathcal{P} ~\ni~ P,Q\; ::=\; K ~~|~~ \alpha.P ~~|~~ \sum_{i\in I} P_i
    ~~|~~ \crn P f \transp{~~|~~ P|Q ~~|~~ P\backslash L}
\end{equation*}
\end{block}

%\setcounter{equation}{0}
\begin{exampleblock}{Exercise: Which are syntactically correct?}
\frsplit[.40]{
\begin{align}
  & a.b.A+B\\&
  (a.\cnil + \ainv a.A) \backslash \set{a,b}\\&
  (a.\cnil + \ainv a.A) \backslash \set{a,\tau}\\&
  a.V+[a/b]\\&
  \tau.\tau.B + \cnil\\&
  (a.B + b.B)[a/b,b/a]
\end{align}
}{\begin{align} &
  (a.B + \tau.B)[a/b,b/a]\\&
  (a.b.A + \ainv a.\cnil)|B\\&
  (a.b.A + \ainv a.\cnil).B\\&
  (a.b.A + \ainv a.\cnil)+B\\&
  (\cnil | \cnil) + \cnil
\end{align}
}
\end{exampleblock}

\end{slide}

%-------------------------------------------------------------------------------

\begin{slide}{CCS semantics - building an LTS}
\small 
\centering
\newcommand{\msep}{~~~~~~}

\typerule{act}{\shrk}{\alpha.P \trans\alpha P}
\msep
\typerule{sum-j}{P_j \trans\alpha P'_j}{\sum_{i\in I}P_i \trans\alpha P'_j} $\transp{j\in I}$
\\[3mm]
\typerule{com1}{P\trans\alpha P'}{P|Q\trans\alpha P'|Q}
\msep
\typerule{com2}{Q\trans\alpha Q'}{P|Q\trans\alpha P|Q'}
\msep
\typerule{com3}{P\trans a P' \quad Q\trans{\ainv{a}} Q'}{P|Q\trans\tau P'|Q'}
\\[3mm]
\typerule{res}{P\trans\alpha P'}{\crt P L\trans\alpha \crt{P'}L}~$\transp{\alpha,\ainv{\alpha} \notin L}$
\msep
\typerule{rel}{P\trans\alpha P'}{\crn P f\trans{f(\alpha)} \crn{P'} d}
\\[2mm]
\pause

\begin{exampleblock}{Exercise: Draw the LTS's}
  \vspace*{-4mm}
  \begin{align*}
    CM &= \mathsf{coin.\ainv{coffee}}.CM
    \\
    CS &= \mathsf{\ainv{pub}.\ainv{coin}.coffee}.CS
    \\
    SmUni &= \crt{(CM|CS)}{\mathsf{\set{coin,coffee}}}
  \end{align*}
\end{exampleblock}

\end{slide}

%-------------------------------------------------------------------------------
\begin{frame}[fragile]{mCRL2}
\small
\myblock{\url{http://mcrl2.org}}
  
\begin{block}{Syntax (by example)}
\begin{align*}
  a.P &\to \mcode{a.P}
  \\
  P_1 + P_2 &\to \mcode{P1 + P2}
  \\
  P\backslash L &\to \mcode{hide(L,P)}
%  \\
%  \crn{P}{...} &\to \mcode{...}
  \\
  a.P|\ainv a.Q  &\to \mcode{hide(\{a\},block(\{a1,a2\},comm(\{a1|a2->a\}, a1.P||a2.P)))}
\end{align*}
\end{block}
\end{frame}

%-------------------------------------------------------------------------------
\begin{frame}[fragile]{mCRL2}
\small
  \myblock{\url{http://mcrl2.org}}
  
%  \begin{block}{Syntax (by example)}
\begin{lstlisting}
act
  coin, coin', coinCom,
  coffee, coffee', coffeeCom, pub';

proc
  CM = coin.coffee'.CM;
  CS = pub'.coin'.coffee.CS;
  CMCS = CM || CS;
  SmUni = hide({coffeeCom,coinCom},
          block({coffee,coffee',coin,coin'},
          comm({coffee|coffee' -> coffeeCom,
                coin|coin'     -> coinCom},
          CMCS )));
init
  SmUni;
\end{lstlisting}
%  \end{block}
\end{frame}

%-------------------------------------------------------------------------------
\section{Behavioural equivalences}

%----------------------------------------------------------------------------------
\begin{slide}{Behavioural Equivalences -- Intuition}
\small


Two LTS should be \dkb{equivalent} if they cannot be distinguished by interacting with them.


\begin{block}{Equality of functional behaviour}
is not preserved by \dgold{parallel} composition: non \dgold{compositional} semantics, cf,
\begin{center}
\texttt{x:=4; x:=x+1} ~and~ \texttt{x:=5}
\end{center}
\end{block}

\begin{block}{Graph isomorphism} 
is too strong (why?)
\end{block}

\end{slide}



%----------------------------------------------------------------------------------
\begin{slide}{Trace}
\small

\begin{block}{Definition}
Let $T = \pair{S, \Act,  \rra}$ be a labelled transition system. The set of \dkb{traces} $\Tr{s}$, for $s \in S$ is the minimal set  satisfying
\begin{align*}
\text{\dgold{(1)}}\; \;  &  \epsilon \in \Tr{s}\\
\text{\dgold{(2)}}\; \;  &  a \sigma \in \Tr{s}\; \imp\; \rcb{\exists}{s'}{s' \in S}{s \rtran{a} s'\, \e\, \sigma \in \Tr{s'}}
\end{align*}
\end{block}


\end{slide}

%----------------------------------------------------------------------------------
\begin{slide}{Trace  equivalence}
\small

\begin{block}{Definition}
Two states $s, r$ are \dkb{trace equivalent} iff  $ \Tr{s} =  \Tr{r}$\\
(i.e. if they can perform the same finite sequences of transitions)
\end{block}
~\\

\begin{block}{Example}
  \centering
  % Requires \usepackage{graphicx}
  \includegraphics[width=9cm]{./images/alarm3.pdf}
\end{block}


\dgold{Trace equivalence} applies  when one can neither interact with a system, nor distinguish a slow system from one that has come to a stand still.
\end{slide}



\section{Similarity}

%----------------------------------------------------------------------------------
\begin{slide}{Simulation}
\begin{flushright}
the quest for a \dkb{behavioural equality}:\\
able to identify states that cannot be distinguished by any \dkb{realistic} form of  observation
\end{flushright}
~\\

\small
\begin{block}{Simulation}
\caixa{A state $q$ \dkb{simulates} another state $p$
\structure{if}\\
every transition from $q$ is corresponded by a transition from $p$
\structure{and}\\
this capacity is kept along the whole life of the system to which state space $q$ belongs to.}
\end{block}
\end{slide}

%----------------------------------------------------------------------------------
\begin{slide}{Simulation}
\small

\begin{block}{Definition}
Given  $\pair{S_1, \Act, \rra_1}$  and $\pair{S_2, \Act, \rra_2}$
 over $\Act$,
relation $R \subseteq S_1 \times S_2$ is a \dkb{simulation} iff,
for all $\pair{p,q} \in R$ and $a \in \Act$,

\begin{align*}
%\text{\dgold{(1)}}\; \;  & p \downarrow_1 \;  \imp\; q \downarrow_2\\ &\\
\text{\dgold{(1)}}\; \;  & p \rtran{a}_1 p'\;  \imp\; \rcb{\exists}{q'}{q' \in S_2}{q \rtran{a}_2 q'\, \e\, \pair{p',q'} \in R}   
\end{align*}
\vspace{6mm}

\begin{equation*}
\xymatrix{
p \ar[d]^-{a} & \hspace{-1.5cm} R  & \hspace{-2.7cm} q 
  & \!\! \text{\raisebox{-10mm}{\Huge $\imp$}} \! \! &  &  &  q \ar[d]^-{a} \\
p'           &   &           &                                   & p'\hspace{-2.7cm} &  R \hspace{-1.5cm} &  q'
}
\end{equation*}

\end{block}
\end{slide}

%----------------------------------------------------------------------------------
\begin{slide}{Example}
\small
\begin{equation*}
\xymatrix{
& q_1  \ar[r]^{d} & q_2 &       &        &                              p_2\\
q_0 \ar[ru]^{a} \ar[rd]_{a} &  & & p_0 \ar[r]^{a} & p_1 \ar[ru]^{d} \ar[rd]_{e} & \\
& q_4  \ar[r]_{e} & q_3 &       &        &                              p_3\\
}
\end{equation*}

\vspace{0.2cm}
\begin{equation*}
q_0 \lesssim p_0 \text{\hspace{0.5cm} cf. \hspace{0.3cm}} 
\enset{\pair{q_0,p_0}, \pair{q_1,p_1},\pair{q_4,p_1},\pair{q_2,p_2},\pair{q_3,p_3}}
\end{equation*}
\end{slide}

%----------------------------------------------------------------------------------
\begin{slide}{Similarity}
\small

\begin{block}{Definition}
\[p \lesssim q\; \equiv\; \rcb{\exists}{R}{}{R\; \text{is a simulation and}\; \pair{p,q} \in R} 
\]
\end{block}
\begin{block}{Lemma}
The similarity relation is a preorder\\
\gry{(ie, reflexive and transitive)}
\end{block}
\end{slide}


\section{Bisimilarity}
%----------------------------------------------------------------------------------
\begin{slide}{Bisimulation}
\small

\begin{block}{Definition}
Given  $\pair{S_1, \Act,  \rra_1}$  and $\pair{S_2, \Act, \rra_2}$ over $\Act$,
relation $R \subseteq S_1 \times S_2$ is a \dkb{bisimulation} iff both $R$ and its converse $\aconv{R}$
are simulations.

I.e.,
whenever $\pair{p,q} \in R$ and $a \in \Act$,

\begin{align*}
%\text{\dgold{(1)}}\; \;  &  p \downarrow_1 \;  \dimp\; q \downarrow_2\\ &\\
\text{\dgold{(1)}}\; \;  & p \rtran{a}_1 p'\; \imp\; \rcb{\exists}{q'}{q' \in S_2}{q \rtran{a}_2 q'\, \e\, \pair{p',q'} \in R}   \\
\text{\dgold{(2)}}\; \;  & q \rtran{a}_2 q'\; \imp\; \rcb{\exists}{p'}{p' \in S_1}{p \rtran{a}_1 p'\, \e\, \pair{p',q'} \in R}   
\end{align*}

%\begin{equation*}
%\xymatrix{
%p \ar[d]^-{a} & \hspace{-1.5cm} R  & \hspace{-2.7cm} q
%    & \! \! \text{\raisebox{-10mm}{\Huge $\Leftrightarrow$}} \! \! &   &  
%    &  q \ar[d]^-{a} \\
%p'           &   &           &                                   & p'\hspace{-2.7cm} &  R \hspace{-1.5cm} &  q'
%}
%\end{equation*}
\centering
\begin{tikzpicture}[%
    every edge/.style={draw, thick,-latex',shorten >= 2pt}]
  \tikzstyle{l} = [auto,inner sep=1pt]
  % the square
  \node(p){$p$};\node[right=2.3 of p](q2){$q$};
  \node[below=1 of p](p'){$p'$};\node(q')at(p'-|q2){$q'$};
  % the R parts
  \node[right=0 of p](R1){$R$};\node[right=0 of R1](q){$q$};
  \node[left=0 of q'](R2){$R$};\node[left=0 of R2](p'2){$p'$};
  % the arrows
  \draw[->] (p)edge node[l]{$a$}(p') (q2)edge node[l]{$a$}(q');
  \node at ($(p)!.5!(q')$){{\Large $\Rightarrow$}};
  %%%%%%%
  % the square
  \node[right=2.3 of q2](pp){$p$};\node[right=2.3 of pp](q2){$q$};
  \node[below=1 of pp](p'){$p'$};\node(q')at(p'-|q2){$q'$};
  % the R parts
  \node[right=0 of p'](R1){$R$};\node[right=0 of R1](q){$q'$};
  \node[left=0 of q2](R2){$R$};\node[left=0 of R2](p'2){$p$};
  % the arrows
  \draw[->] (pp)edge node[l]{$a$}(p') (q2)edge node[l]{$a$}(q');
  \node at ($(pp)!.5!(q')$){{\Large $\Leftarrow$}};

\end{tikzpicture}

\end{block}
\end{slide}



%----------------------------------------------------------------------------------
\begin{slide}{Examples}
\small
\begin{equation*}
\xymatrix{
& q_1  \ar[ld]_{a}  \ar[rd]^{a} & & & & m \ar[d]_{a}\\
q_2 \ar[rr]^{c}  && q_3 \ar@(ur,dr)[]^{c}  & & & n\ar@(ur,dr)[]^{c}
}
\end{equation*}
% R = \enset{(q_1,m), (q_2, n), (q_3,n) }
\vspace{1cm}

\begin{equation*}
\xymatrix{
q_1 \ar[r]^{a} & q_2  \ar[r]^{a}  & q_3  \ar[r]^{a} & \cdots & & h\ar@(ur,dr)[]^{a}  \\
}
\end{equation*}
% R = \setdef{q_i, h}{i \geq 1}
\end{slide}


%----------------------------------------------------------------------------------
\begin{slide}{Examples}
\small
\begin{equation*}
\xymatrix{
& q_1  \ar[ld]_{a}  \ar[rd]^{a} & & & & p_1  \ar[d]^{a} \\
q_2 \ar[d]^{c}  && q_3  \ar[d]^{c}  & && p_2  \ar[ld]_{c}  \ar[rd]^{c} & \\
q_4 & & q_5 & &p_4 & & p_5\\
& q_1  \ar[ld]_{a}  \ar[rd]^{a} & & & & p_1  \ar[d]^{a} \\
q_2 \ar[d]^{c}  && q_3  \ar[d]^{\red b }  & && p_2  \ar[ld]_{c}  \ar[rd]^{\red b } & \\
q_4 & & q_5 & &p_4 & & p_5
}
\end{equation*}
\end{slide}

%----------------------------------------------------------------------------------
\begin{slide}{After thoughts}
\small


\begin{itemize}
\item Follows a \dgold{$\forall, \exists$ pattern}: $p$ in all its transitions challenge $q$ which is called to find a matchh to each of those (and conversely)
\item Tighter correspondence with transitions
\item Based on the information that the transitions convey, rather than on the shape of the LTS
\item \dgold{Local} checks on states
\item \dgold{Lack of hierarchy}  on the pairs of the bisimulation (no temporal order on the checks is required)
\end{itemize}
which means \dgold{bisimilarity can be used to reason about infinite or circular behaviours}.
\end{slide}

%----------------------------------------------------------------------------------
\begin{slide}{After thoughts}
\small
~\\

\noindent
Compare the definition of bisimilarity with 
~\\
~\\


$p == q$ if, for all $a  \in \Act$
\begin{align*}
%\text{\dgold{(1)}}\; \;  &  p \downarrow_1 \;  \dimp\; q \downarrow_2\\ &\\
\text{\dgold{(1)}}\; \;  & p \rtran{a}_1 p'\; \imp\; \rcb{\exists}{q'}{q' \in S_2}{q \rtran{a}_2 q'\, \e\, p' == q'}   \\
\text{\dgold{(2)}}\; \;  & q \rtran{a}_2 q'\; \imp\; \rcb{\exists}{p'}{p' \in S_1}{p \rtran{a}_1 p'\, \e\, p' == q'}    
\end{align*}

\end{slide}

%----------------------------------------------------------------------------------
\begin{slide}{After thoughts}
\small
~\\


$p == q$ if, for all $a  \in \Act$
\begin{align*}
%\text{\dgold{(1)}}\; \;  &  p \downarrow_1 \;  \dimp\; q \downarrow_2\\ &\\
\text{\dgold{(1)}}\; \;  & p \rtran{a}_1 p'\; \imp\; \rcb{\exists}{q'}{q' \in S_2}{q \rtran{a}_2 q'\, \e\, p' == q'}   \\
\text{\dgold{(2)}}\; \;  & q \rtran{a}_2 q'\; \imp\; \rcb{\exists}{p'}{p' \in S_1}{p \rtran{a}_1 p'\, \e\, p' == q'}    
\end{align*}

\gry{
\begin{itemize}
\item The meaning of $==$ on the pair $\pair{p,q}$ requires having already established the meaning of $==$ on the derivatives
\item ... therefore the definition is \dgold{ill-founded} if the state space reachable from $\pair{p,q}$ is infinite or contain loops
\item ... this is a \dgold{local} but \dgold{inherently inductive} definition (to revisit later)
\end{itemize}
}
\end{slide}



%----------------------------------------------------------------------------------
\begin{slide}{After thoughts}
\small

\begin{block}{Proof method}

To prove that two behaviours are bisimilar, find a bisimulation containing them ...



\begin{itemize}
\item
... \dgold{impredicative} character
\item \dgold{coinductive} vs \dgold{inductive} definition
\end{itemize}
\end{block}
\end{slide}

%----------------------------------------------------------------------------------
\begin{slide}{Properties}
\small

\begin{block}{Definition}
\[p \sim q\; \equiv\; \rcb{\exists}{R}{}{R\; \text{is a bisimulation and}\; \pair{p,q} \in R} 
\]
\end{block}
\begin{block}{Lemma}
\begin{enumerate}
\item The identity relation $\id$ is a bisimulation
\item The empty relation $\bot$ is a bisimulation
\item The converse $\aconv{R}$ of a bisimulation is a bisimulation
\item The composition $S \comp R$ of two bisimulations $S$ and $R$ is a bisimulation
\item The $\bigcup_{i \in I} R_i$ of a family of bisimulations
 $\setdef{R_i}{i \in I}$ is a bisimulation
\end{enumerate}
\end{block}
\end{slide}

%----------------------------------------------------------------------------------
\begin{slide}{Properties}

\small


\begin{block}{Lemma}
The bisimilarity relation is an equivalence relation\\
\gry{(ie, reflexive, symmetric and transitive)}
\end{block}

\begin{block}{Lemma}
The class of all bisimulations between two LTS has the structure of a 
\dkb{complete lattice}, ordered by set inclusion, whose top is the  \dkb{bisimilarity} relation $\sim$.
\end{block}
\end{slide}

%----------------------------------------------------------------------------------
\begin{slide}{Properties}

\small


\begin{block}{Lemma}
In a \dgold{deterministic} labelled transition system, two states are bisimilar iff they are trace equivalent, i.e.,
$$
s \obs s'\; \; \dimp\;\; \Tr{s} = \Tr{s'}
$$
~\\

\gry{Hint: define a relation $R$ as 
$$\pair{x,y} \in R \; \dimp\;  \Tr{x} = \Tr{y}$$
 and show $R$ is a bisimulation.}
\end{block}


\end{slide}

%----------------------------------------------------------------------------------
\begin{slide}{Properties}

\small

\begin{alertblock}{Warning}
\centering
\fbox{The bisimilarity relation \dkb{$\sim$} {\red is not} the \dkb{symmetric closure of $\lesssim$}}
\end{alertblock}

\begin{block}{Example}
\begin{center}
$q_0 \lesssim p_0,\;  p_0 \lesssim q_0\;\;\; \text{but}\;\;\;  p_0 \not \sim q_0$
\end{center}

\begin{equation*}
\xymatrix{
& q_1  & & & & \\
q_0 \ar[ru]^{a} \ar[rd]^{a} &  & & p_0 \ar[r]^{a} & p_1 \ar[r]^{b} & p_3\\
& q_2  \ar[r]^{b} & q_3 &       &        &                          
}
\end{equation*}
\end{block}
\end{slide}




%----------------------------------------------------------------------------------
\begin{slide}{Notes}
\small
\begin{block}{Similarity as the \underline{greatest} simulation}
\begin{equation*}
\dkb{\lesssim}\;  \deff\; \bigcup \setdef{S}{S\, \text{is a simulation}} 
\end{equation*}
\end{block}
\begin{block}{Bisimilarity as the \underline{greatest} bisimulation}
\begin{equation*}
\dkb{\sim}\;  \deff\; \bigcup \setdef{S}{S\, \text{is a bisimulation}} 
\end{equation*}
\end{block}
%\begin{flushright}
%cf \dgold{relational} translation of definitions\\
%$\dkb{\lesssim}$ and $\dkb{\sim}$ as \dgold{greatest fix points} (Tarski's theorem)
%\end{flushright}
\end{slide}

%-------------------------------------------------------------------------------

\begin{slide}{Exercises}
\frsplit{
\begin{exampleblock}{P,Q Bisimilar?}
 \vspace*{-4mm}
 \begin{align*}
  P&=a.P_1\\
  P_1&=b.P + c.P\\[3mm]
  Q&=a.Q_1\\
  Q_1&=b.Q_2+c.Q\\
  Q_2&=a.Q_3\\
  Q_3&=b.Q+c.Q_2
 \end{align*} 
\end{exampleblock}
}{
\begin{exampleblock}{P,Q Bisimilar?}
 \vspace*{-4mm}
 \begin{align*}
  P&=a.(b.\cnil+c.\cnil)\\[3mm]
  Q&=a.b.\cnil + a.c.\cnil
 \end{align*}  
\end{exampleblock}
}
\end{slide}

%-------------------------------------------------------------------------------

\begin{slide}{Exercises}
\begin{exampleblock}{Find a bisimulation}
%\vspace*{-3mm}
\frsplit{\centering
  \includegraphics{images/LTS-1}
}{\centering
  \includegraphics{images/LTS-2}
}
\end{exampleblock}
\end{slide}

%-------------------------------------------------------------------------------

\begin{slide}{mCRL2}
  
\begin{itemize}
  \item {Formal \dkb{specification language} with an associated toolset}
  
  \item {Used for \dkb{modelling}, \dkb{validating} and \dkb{verifying} concurrent systems and protocols}
\end{itemize}
  
\end{slide}

%-------------------------------------------------------------------------------

\begin{slide}{mCRL2 - toolset overview}
\centering  

\includegraphics[width=.9\textwidth]{images/mcrl2-toolset}
  
\end{slide}

%-------------------------------------------------------------------------------

\end{document}

