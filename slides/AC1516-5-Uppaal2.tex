%\documentclass[handout]{beamer}
\documentclass{beamer}
\usepackage{etex} % fixes new-dimension error

\definecolor{BrickRed}{rgb}{0.8, 0.25, 0.33}

%\usetheme{CambridgeUS}%{Copenhagen}%{Frankfurt}%{Singapore}%{CambridgeUS}
\usecolortheme[named=BrickRed]{structure} 
\useoutertheme[subsection=false]{miniframes}
\useinnertheme{circles}
%\useinnertheme[shadow=false]{rounded}
\setbeamertemplate{blocks}[rounded][shadow=false]

\setbeamercovered{transparent} 
\setbeamertemplate{navigation symbols}{} %Remove navigation bar
\setbeamertemplate{footline}[frame number] % add page number
\setbeamercolor{postit}{fg=BrickRed,bg=red!10}
\setbeamercolor{structure}{bg=black!10}
%\setbeamercolor{palette primary}{use=structure,fg=red,bg=green}
%\setbeamercolor{palette secondary}{use=structure,fg=red!75!black,bg=green}
\setbeamercolor{palette tertiary}{use=structure,bg=red!50!black,fg=white}
%\setbeamercolor{palette quaternary}{fg=black,bg=green}
%\setbeamercolor{normal text}{fg=black,bg=white}
%\setbeamercolor{block title alerted}{fg=red,bg=green}
%\setbeamercolor{block title example}{bg=black!10,fg=green}
\setbeamercolor{block body}{bg= black!5}

\setbeamercolor{block title alerted}{bg=red!25}
\setbeamercolor{block body alerted}{bg=red!10}

\setbeamercolor{block title example}{bg={rgb:green,2;black,1;white,5}}
\setbeamercolor{block body example}{bg={rgb:green,2;black,1;white,20}}


%----------------------------------------------------------------------------
\usepackage{graphicx,amsmath}
\usepackage{stmaryrd} % cf. interleave
\usepackage{./macros/myisolatin1}
%\usepackage{/Users/lsb/Library/tex/latex/prooftree}
\usepackage{amscd}
%------ using xy ------------------------------------------------------------
\usepackage{alltt}
%------ using xy ------------------------------------------------------------
\usepackage[all]{xy}
%\def\larrow#1#2#3{\xymatrix{ #3 & #1 \ar[l] _-{#2} }}
\def\larrow#1#2#3{\xymatrix{ #3 & #1 \ar[l] _--{#2} }}
\def\rarrow#1#2#3{\xymatrix{ #1 \ar[r]^-{#2} & #3 }}
\def\arLaw#1#2#3#4#5{
\xymatrix{
        #1      \ar@/^1pc/[rr]^-{#4} &
        #5 &
        #2      \ar@/^1pc/[ll]^-{#3}
}}
\def\arLeq#1#2#3#4{\arLaw{#1}{#2}{#3}{#4}\leq}
%------ using pstricks (rnode etc) ------------------------------------------
\usepackage{pstricks,pst-node,pst-text,pst-3d}
%------ using color ---------------------------------------------------------
\newrgbcolor{goldenrod}{.80392 .60784 .11373}
\newrgbcolor{darkgoldenrod}{.5451 .39608 .03137}
\newrgbcolor{brown}{.15 .15 .15}
\newrgbcolor{darkolivegreen}{.33333 .41961 .18431}
%
%
\def\gold#1{{\goldenrod #1}}
\def\dgold#1{{\darkgoldenrod #1}}
%\def\brw#1{{\brown #1}}
\def\dkb#1{{\blue #1}}
\def\tdkb#1{\textbf{\darkblue #1}}
%%\def\gre#1{{\green #1}}
\def\gre#1{{\darkolivegreen #1}}
\def\gry#1{{\gray #1}}
\def\rdb#1{{\red #1}}
\def\st{\mathbf{.}\,}
\def\laplace#1#2{*\txt{\mbox{ \fcolorbox{black}{myGray}{$\begin{array}{c}\mbox{#1}\\\\#2\\\\\end{array}$} }}}
%\newcommand{\galois}[2]{#1\; \dashv\; #2}

\def\eqm{\mathbin{\equiv}}                     
\def\noeqm{\mathbin{\not\!\equiv}}  
%\newcommand{\flam}[2]{\lambda_{#1}\; .\; #2}
\def\existential#1#2{\exists_{#1}\;.\; #2}
\def\existencial#1#2{\exists_{#1}\;.\; #2}

\def\pv#1#2{\langle #1 \rangle #2}
\def\nc#1#2{[#1]#2}
\def\pvo#1#2{\langle \! \! \! \langle #1 \rangle \! \! \! \rangle\, #2}
\def\nco#1#2{\llbracket #1 \rrbracket #2}
\def\cvg#1{\llbracket \downarrow \rrbracket #1}
\def\cvgr#1#2{\llbracket #1 \downarrow \rrbracket #2}
\def\cvgl#1#2{\llbracket \downarrow  #1 \rrbracket #2}
\def\cvglr#1#2{\llbracket \downarrow  #1 \downarrow \rrbracket #2}
\def\lfp#1#2{\mu {#1}\, .\, {#2}}
\def\lpf#1#2{\mu {#1}\, .\, {#2}}
\def\gfp#1#2{\nu {#1}\, .\, {#2}}
\def\gpf#1#2{\nu {#1}\, .\, {#2}}
\def\mset#1{\vvv #1 \vvv}
\def\vvv{\vert \! \vert}
\def\mnc#1{\vvv [#1] \vvv}
\def\mpv#1{\vvv \langle #1 \rangle \vvv}
\def\bcomp#1{#1^{\text{c}}}
\def\eqm{\mathbin{\simeq}}
\def\noeqm{\mathbin{\not\!\simeq}}
\def\universal#1#2{\forall_{#1}\;.\; #2}
\def\existential#1#2{\exists_{#1}\;.\; #2}
\def\oexistential#1#2{\exists^{1}_{#1}\;.\; #2}
\def\MM{\mathcal{M}}
\def\uppaal{\textsc{Uppaal}}
\def\cc#1{\mathcal{C}(#1)}
\def\R{\mathcal{R}}
\def\TL#1{\mathcal{T}(#1)}
\def\ET#1{\mathsf{ExecTime(#1)}}


\input{./macros/LSBslides-mac}
%\input{LSBslides1.tex}
\title{
	Time-critical reactive systems\\ (verification)	}
\author{Jos\'e Proen\c{c}a}
\institute{HASLab - INESC TEC \\ Universidade do Minho\\ Braga, Portugal}

\date{
\begin{tabular}{c}
\\
  May  2016
\end{tabular}

}

\begin{document}

\frame[plain]{\titlepage}




\section{Behavioural equivalences}

%----------------------------------------------------------------------------------
\begin{slide}{Traces}
\small

\begin{block}{Definition}
A \dkb{timed trace} over a \dgold{temporal LTS} is a (finite or infinite) sequence  $\pair{t_1,a_1}, \pair{t_2,a_2}, \cdots$ in
 $\R^+ \times Act$ such that there exists a path
\begin{equation*}
\pair{l_0,\eta_0}  \tran{d_1}   \pair{l_0,\eta_1}    \tran{a_1}     \pair{l_1,\eta_2}    \tran{d_2}    \pair{l_1,\eta_3}   \tran{a_2} \cdots 
\end{equation*}
such that 
\begin{equation*}
t_{i} = t_{i-1} + d_i
\end{equation*}
with $t_0=0$ and, for all clock $x$, $\eta_0\, x = 0$.
\end{block}
~\\

Intuitively, each $t_i$ is an absolute time value acting as a \dgold{time-stamp}.

\begin{alertblock}{Warning}
All results from now on are given over an arbitrary \dgold{temporal LTS}; they naturally apply to $\TL{ta}$ for any timed automata $ta$.
\end{alertblock}
\end{slide}

%----------------------------------------------------------------------------------
\begin{slide}{Traces}
\small

Given a \dgold{timed trace} $tc$, the corresponding \dkb{untimed trace} is $(\pi_2)^{\omega}\,  tc$.

\begin{block}{Definition}
\begin{itemize}
\item two states $s_1$ and $s_2$ of a timed LTS are \dkb{timed-language equivalent} if the \dgold{set of finite timed traces}
of  $s_1$ and $s_2$ coincide;
\item ... similar definition for \dkb{untimed-language equivalent} ...
\end{itemize}
\end{block}

\begin{block}{Example}
\begin{tabular}{cc}
 \includegraphics[width=2cm]{./images/tr1.jpg} &   %\hspace{1cm}
  \includegraphics[width=2cm]{./images/tr2.jpg}
\end{tabular}
are not \dkb{timed-language equivalent}: $\pair{(0,t)}$ is not a trace of the TLTS generated by the second system.
\end{block}
\end{slide}

%----------------------------------------------------------------------------------
\begin{slide}{Bisimulation}
\small

\begin{block}{Timed bisimulation}
A relation $R$ is a \dkb{timed simulation} iff whenever $s_1 R s_2$, for any action $a$ and delay $d$,
\begin{align*}
s_1\tran{a} s_1'\; & \imp\; \text{there is a transition}\; \; \; s_2 \tran{a} s_2' \e s_1' R s_2'\\
s_1 \tran{d} s_1'\; & \imp\; \text{there is a transition}\; \; \; s_2 \tran{d} s_2' \e s_1' R s_2'
\end{align*}
And a \dkb{timed bisimulation} if its converse is also a bisimulation.
\end{block}
\end{slide}

%----------------------------------------------------------------------------------
\begin{slide}{Bisimulation}
\small

\begin{block}{Example}

\includegraphics[width=6cm]{./images/bis1.jpg} 
\end{block}

\begin{align*}
\pair{\pair{W1,[x=0]}, \pair{Z1,[x=0]} } \in R
\end{align*}
where
\begin{align*}
R \; =\;  & \setdef{\pair{\pair{W1,[x=d]}, \pair{Z1,[x=d]} } }{d \in \R_0^+}\; \;   \cup\\
              & \setdef{\pair{\pair{W2,[x=d+1]}, \pair{Z2,[x=d]} } }{d \in \R_0^+}\; \;  \cup\\
              & \setdef{\pair{\pair{W3,[x=d]}, \pair{Z3,[x=e]} } }{d,e \in \R_0^+}  
\end{align*}


\end{slide}



%----------------------------------------------------------------------------------
\begin{slide}{Bisimulation}
\small

\begin{block}{Untimed bisimulation}
A relation $R$ is a \dkb{untimed simulation} iff whenever $s_1 R s_2$, for any action $a$ and delay $t$,
\begin{align*}
s_1 \tran{a} s_1' \; & \imp\; \text{there is a transition}\; \; \; s_2 \tran{a} s_2' \e s_1' R s_2'\\
s_1 \tran{d} s_1'\; & \imp\; \text{there is a transition}\; \; \; s_2 \tran{d'} s_2' \e s_1' R s_2'
\end{align*}
And a \dkb{untimed bisimulation} if its converse is also a  untimed bisimulation.
\end{block}
~\\


\dgold{Alternatively, it can be defined over a modified LTS in which all delays are abstracted on a 
unique, special transition labelled by $\epsilon$.}
\end{slide}



\section{Behavioural properties}
%----------------------------------------------------------------------------------
\begin{slide}{Properties: expression and satisfaction}
\small
\begin{block}{The satisfaction problem }
Given a \dgold{timed automata}, $ta$, and a \dkb{property}, $\phi$, show that
\begin{equation*}
\TL{ta} \models \phi
\end{equation*}
\end{block}
~\\

\pause
\begin{itemize}
\item in which logic language shall $\phi$ be specified?
\item how is $\models$ defined?
\end{itemize}
\end{slide}




%----------------------------------------------------------------------------------
\begin{slide}{Expressing properties: \textsc{Uppaal}}
\small

\begin{block}{\textsc{Uppaal} variant of \textsc{Ctl}}
\begin{itemize}
\item \dkb{state formulae}:  describes individual states in $\TL{ta}$
\item \dkb{path formulae}: describes properties of paths in $\TL{ta}$
\end{itemize}
\end{block}

\end{slide}

%----------------------------------------------------------------------------------
\begin{slide}{Expressing properties: \textsc{Uppaal}}
\small

\begin{block}{State formulae}
Any expression which can be evaluated to a boolean value for a state (typically involving the 
\dgold{clock constraints} used for guards and invariants and similar constraints over integer variables):
\begin{center}
\texttt{x >= 8}, \texttt{i == 8 and x < 2}, ...
\end{center}
Additionally,
\begin{itemize}
\item $ta.\texttt{l}$ which tests \dgold{current location}:  $(l, \eta) \models ta.\texttt{l}$ \\
provided $(l, \eta)$ is a state in $\TL{ta}$
\item $\texttt{deadlock}$: $(l, \eta) \models \forall_{d \in \R_0^+} .\, \text{there is no transition from} \; \pair{l,\eta+d}$ 
\end{itemize}
\end{block}

\end{slide}

%----------------------------------------------------------------------------------
\begin{slide}{Expressing properties: \textsc{Uppaal}}
\small

\begin{block}{Path formulae}
\begin{align*}
\Pi\; ::=\; & A \Box\, \Psi\, \mid\, A\Diamond\, \Psi\, \mid\, E \Box\, \Psi\, \mid\, E \Diamond\, \Psi\, \mid\, \Phi \leadsto  \Psi\
\end{align*}

where
\begin{itemize}
\item \dkb{$A$, $E$} quantify (universally and existentially, resp.) over paths
\item \dkb{$\Box$, $\Diamond$} quantify (universally and existentially, resp.) over states in a path
\end{itemize}
also notice that

\begin{align*}
 \Phi \leadsto  \Psi\; \abv\; A \Box\, (\Phi \imp E \Diamond\, \Psi)
\end{align*}
\end{block}

\end{slide}

%----------------------------------------------------------------------------------
\begin{slide}{Expressing properties: \textsc{Uppaal}}
\small

\begin{block}{$A \Box\, \varphi$ and $A \Diamond \, \varphi$}
\begin{tabular}{cc}
 \includegraphics[width=3cm]{./images/AA.jpg} &   \hspace{1cm} \includegraphics[width=3cm]{./images/AE.jpg}
\end{tabular}
\end{block}

\begin{block}{$E \Box\, \varphi$ and $E \Diamond\, \varphi$}
\begin{tabular}{cc}
 \includegraphics[width=3cm]{./images/EA.jpg} &   \hspace{1cm} \includegraphics[width=3cm]{./images/EE.jpg}
\end{tabular}
\end{block}
\end{slide}

%----------------------------------------------------------------------------------
\begin{slide}{Expressing properties: \textsc{Uppaal}}
\small

\begin{block}{$\varphi\, \leadsto\, \psi$}
\begin{center}
 \includegraphics[width=5cm]{./images/LeadsTo.jpg} 
 \end{center}
\end{block}

\end{slide}

%----------------------------------------------------------------------------------
\begin{slide}{Reachability properties}
\small

\begin{block}{$E \Diamond\, \phi$}
\dkb{Is there a path starting at the initial state, such that a state formula $\phi$ is eventually satisfied?}

\begin{itemize}
\item  Often used to perform sanity checks  on a model:
\begin{itemize}
\item \dgold{is it possible for a sender to send a message?}
 \item \dgold{can a message possibly be received?}
 \item ...
 \end{itemize}
 \item  Do not by themselves guarantee the correctness of the protocol (i.e. \dgold{that any message is eventually delivered}), 
but they validate the basic behavior of the model.
 \end{itemize}
\end{block}

\end{slide}


%----------------------------------------------------------------------------------
\begin{slide}{Safety properties}
\small

\begin{block}{$A \Box\, \phi$ and $E \Box\, \phi$}
\vspace{5mm}
\dkb{Something bad will never happen}\\
 or \dkb{something bad will possibly never happen}
\vspace{5mm}

Examples\\
\begin{itemize}
\item \dgold{In a nuclear power plant the temperature of the core is always (invariantly) under a certain threshold}.
\item \dgold{In a game a safe state is one in which we can still win, ie, will possibly not loose}.
\end{itemize}

In Uppaal these properties are formulated positively: something good is invariantly true.
\end{block}

\end{slide}

%----------------------------------------------------------------------------------
\begin{slide}{Liveness properties}
\small

\begin{block}{$A \Diamond\, \phi$ and $\phi\, \leadsto \, \psi$}
\vspace{5mm}
\dkb{Something good will eventually happen}\\
 or \dkb{if something good happen, then something else will eventually happen!}
\vspace{5mm}

Examples\\
\begin{itemize}
\item \dgold{When pressing the on button, then eventually the television should turn on}.
\item \dgold{In a communication protocol, any message that has been sent should eventually be received}.
\end{itemize}

\end{block}

\end{slide}



%\end{document}



\section{Case-study: proving mutual exclusion}



%----------------------------------------------------------------------------------
\begin{slide}{The train gate example}
\small

\begin{center}
\includegraphics[width=4cm]{./images/tg1.jpg} 
\end{center}

\begin{itemize}
\item \texttt{E<> Train(0).Cross} 
\\ \dkb{(Train 0 can reach the cross)}
\item \texttt{E<> Train(0).Cross and Train(1).Stop}
\\ \dkb{(Train 0 can be crossing bridge while Train 1 is waiting to cross)}
\item \texttt{E<> Train(0).Cross and (forall (i:id-t) i != 0 imply Train(i).Stop)}
\\ \dkb{(Train 0 can cross bridge while the other trains are waiting to cross)}
\end{itemize}

\end{slide}

%----------------------------------------------------------------------------------
\begin{slide}{The train gate example}
\small

\begin{center}
\includegraphics[width=3cm]{./images/tg1.jpg} 
\end{center}


\begin{itemize}
\item \texttt{A[] Gate.list[N] == 0}
\\ \dkb{There can never be N elements in the queue}
\item \texttt{A[] forall (i:id-t) forall (j:id-t) Train(i).Cross \&\& Train(j).Cross imply i == j}
\\ \dkb{There is never more than one train crossing the bridge}
\item \texttt{Train(1).Appr --> Train(1).Cross}
\\ \dkb{Whenever a train approaches the bridge, it will eventually cross}
\item \texttt{A[] not deadlock}
\\ \dkb{The system is deadlock-free}
\end{itemize}


\end{slide}



%%----------------------------------------------------------------------------------
%\begin{slide}{The train gate example}
%\small
%
%\begin{center}
%\includegraphics[width=5cm]{./images/tg2.jpg} 
%\end{center}
%
%\begin{itemize}
%\item Note the use of parameters and the select clause on transitions
%\item Programming ...
%\end{itemize}
%
%\end{slide}





%%----------------------------------------------------------------------------------
%\begin{slide}{Demo}
%\small
%
%
%\begin{itemize}
%\item The \dgold{train gate} case study (included in the \textsc{Uppaal} distribution).
%\end{itemize}
%
%\end{slide}
%
%




%----------------------------------------------------------------------------------
\begin{slide}{Mutual exclusion}
\small

\begin{block}{Properties}
\begin{itemize}
\item \dkb{mutual exclusion}: \dgold{no two processes are in their critical sections at the same time}
\item \dkb{deadlock freedom}: \dgold{if some process is trying to access its critical section, then 
eventually some process (not necessarily the same) will be in its critical section; similarly for exiting the critical section}
\end{itemize}
\end{block}
\end{slide}

%%----------------------------------------------------------------------------------
\begin{slide}{Mutual exclusion}
\small


\begin{block}{The Problem}
\begin{itemize}
\item 
Dijkstra's original asynchronous algorithm (1965) requires, for $n$ processes to be controlled,
$\mathcal{O}(n)$ read-write registers and $\mathcal{O}(n)$ operations.
\item
This result is a theoretical limit (proved by Lynch and Shavit in 1992) which compromises scalability.
\end{itemize}
\end{block}
\pause
\fbox{but it can be overcome by introducing specific \dkb{timing constraints}}
%\end{block}

\begin{block}{Two \emph{timed} algorithms:}
\begin{itemize}
\item  \dkb{Fisher's protocol} (included in the \textsc{Uppaal} distribution)
\item  \dkb{Lamport's protocol}
\end{itemize}
\end{block}
\end{slide}


%----------------------------------------------------------------------------------
\begin{slide}{Fisher's algorithm}
\small

\begin{block}{The algorithm}
\begin{align*}
\mathsf{repeat} & \\
& \mathsf{repeat}  \\
&  \hspace{1cm} \mathsf{await} \; id = 0\\
& \hspace{1cm} id := i\\
& \hspace{1cm}  \mathsf{delay}(k)\\ 
& \mathsf{until}\; id=i  \\
& \text{\emph{(critical section)}}\\
& id := 0\\
\mathsf{forever} &
\end{align*}
\end{block}
\end{slide}


%%----------------------------------------------------------------------------------
%\begin{slide}{Alur \& Taubenfeld's algorithm}
%\small
%
%\begin{block}{The algorithm}
%\begin{align*}
%\mathsf{repeat} & \\
%& s:\;  x:=i\\
%&\mathsf{await}\, (y = 0)\\
%& y := i\\
%& \mathsf{if}\;  x \neq i \; \mathsf{then}\,\\
%&  \hspace{1cm} \mathsf{delay}(2d); \\
%&  \hspace{1cm}  \mathsf{if}\; y \neq i \; \mathsf{then}\,\mathsf{goto}\, s; \\
%&  \hspace{1cm}  \mathsf{await}\, (\neg z)\\
%&  \mathsf{else}\, z:= true;\\
%& \text{\emph{(critical section)}}\\
%& z := false\\
%& \mathsf{if}\;  y = i \; \mathsf{then}\; y:=0;\\
%\mathsf{forever} &
%\end{align*}
%\end{block}
%\end{slide}


%----------------------------------------------------------------------------------
\begin{slide}{Fisher's algorithm}
\small
\begin{block}{Comments}
\begin{itemize}
\item One shared read/write register (the variable $id$)
\item Behaviour depends crucially on the value for $k$ --- the \dkb{time delay}
\item Constant $k$ should be \dgold{larger than the longest time that a process may take to perform a step while trying to get access to its critical section} 
\item This choice guarantees that whenever process $i$ finds $id = i$ on testing the loop guard it can enter safely ist critical section: \dgold{all} other processes are out of the loop or with their index in $id$ overwritten by $i$.
\end{itemize}
\end{block}
\end{slide}

%----------------------------------------------------------------------------------
\begin{slide}{Fisher's algorithm in \textsc{Uppaal}}
\small

\includegraphics[width=6cm]{./images/ficher1.jpg} 


\begin{itemize}
\item Each process uses a local clock $x$ to guarantee that the upper bound between between its successive steps, while
trying to access the critical section, is $k$ (cf. \dgold{invariant} in state $req$).
\item \dgold{Invariant} in state $req$ establishes $k$ as such an upper bound
\item \dgold{Guard} in transition from $wait$ to $cs$ ensures the correct delay before entering the critical section
\end{itemize}
\end{slide}

%----------------------------------------------------------------------------------
\begin{slide}{Fisher's algorithm in \textsc{Uppaal}}
\small

\begin{block}{Properties}
\includegraphics[width=12cm]{./images/ficher2.jpg} 
\end{block}

\begin{itemize}
\item The algorithm is \dgold{deadlock-free}
\item It ensures  mutual exclusion if the correct timing constraints. 
\item ... but it is critically sensible to  small violations of such constraints: for example, replacing $x > k$ by 
$x \geq k$ in the transition leading to $cs$ compromises both \dgold{mutual exclusion} and \dgold{liveness}.
\end{itemize}
\end{slide}


%----------------------------------------------------------------------------------
\begin{slide}{Lamport's algorithm}
\small

\begin{block}{The algorithm}
\begin{align*}
\mathsf{start}: \; \;  &  a := i\\
& \mathsf{if}\; b \neq 0\; \mathsf{then\; goto\; start}\\
& b := i \\
& \mathsf{if}\; a \neq i\; \mathsf{then\; delay}(k)\\
& \hspace{12mm} \mathsf{else} \; \mathsf{if}\; b \neq i\; \mathsf{then\; goto\; start}\\
& \text{\emph{(critical section)}}\\
& b := 0
\end{align*}
\end{block}
\end{slide}

%----------------------------------------------------------------------------------
\begin{slide}{Lamport's algorithm}
\small
\begin{block}{Comments}
\begin{itemize}
\item Two shared read/write registers (variables $a$ and $b$)
\item Avoids \dgold{forced waiting} when no other processes are requiring access to their critical sections
\end{itemize}
\end{block}
\end{slide}


%----------------------------------------------------------------------------------
\begin{slide}{Lamport's algorithm in \textsc{Uppaal}}
\small

\begin{center}
\includegraphics[width=10cm]{./images/lamport.jpg} 
\end{center}


\end{slide}


%----------------------------------------------------------------------------------
\begin{slide}{Lamport's algorithm}
\small
\begin{block}{Model time constants:}
\begin{description}
\item \dkb{$k$} --- time delay
\item \dkb{$kvr$} --- max bound for register access
\item \dkb{$kcs$} --- max bound for permanence in critical section 
\end{description}
Typically
\begin{center}
\fbox{$k\; \geq \; kvr + kcs$}
\end{center}
\end{block}
\vspace{-5mm}

\begin{block}{Experiments}
\begin{tabular}{|l|c|c|c|c|}
\hline
& $k$ & $kvr$ & $kcs$ & verified? \\
\hline
Mutual Exclusion & 4 & 1 & 1 & Yes\\
Mutual Exclusion & 2 & 1 & 1 & Yes\\
Mutual Exclusion & 1 & 1 & 1 & No\\
No deadlock & 4 & 1 & 1 & Yes\\
No deadlock & 2 & 1 & 1 & Yes\\
No deadlock  & 1 & 1 & 1 & Yes\\
\hline
\end{tabular}
\end{block}
\end{slide}

%%----------------------------------------------------------------------------------
%\begin{slide}{Reading suggestions}
%\small
%
%\begin{center}
% \includegraphics[height=9cm]{./images/lamportPaper.jpg} 
%\end{center}
%
%\end{slide}
%
%%----------------------------------------------------------------------------------
%\begin{slide}{Reading suggestions}
%\small
%
%
%   \includegraphics[height=8cm]{./images/lamportON.jpg}
%
%\end{slide}
%
%
\end{document}

